\input{input_accessability.tex}
\documentclass[11pt,
  english,
  letterpaper,
]{article}
\usepackage{sa4ss}
\usepackage{amsmath,amssymb,array}
\usepackage{booktabs}

% From tagged-template.latex
\usepackage{lmodern}
\usepackage{ifxetex,ifluatex}
\ifnum 0\ifxetex 1\fi\ifluatex 1\fi=0 % if pdftex
  \usepackage[T1]{fontenc}
  \usepackage[utf8]{inputenc}
  \usepackage{textcomp} % provide euro and other symbols
\else % if luatex or xetex
  \usepackage{unicode-math}
  \defaultfontfeatures{Scale=MatchLowercase}
  \defaultfontfeatures[\rmfamily]{Ligatures=TeX,Scale=1}
\fi

% Use upquote if available, for straight quotes in verbatim environments
\IfFileExists{upquote.sty}{\usepackage{upquote}}{}
\IfFileExists{microtype.sty}{% use microtype if available
  \usepackage[]{microtype}
  \UseMicrotypeSet[protrusion]{basicmath} % disable protrusion for tt fonts
}{}
\makeatletter
\@ifundefined{KOMAClassName}{% if non-KOMA class
  \IfFileExists{parskip.sty}{%
    \usepackage{parskip}
  }{% else
    \setlength{\parindent}{0pt}
    \setlength{\parskip}{6pt plus 2pt minus 1pt}}
}{% if KOMA class
  \KOMAoptions{parskip=half}}
\makeatother
\usepackage{xcolor}
\IfFileExists{xurl.sty}{\usepackage{xurl}}{} % add URL line breaks if available
\hypersetup{
  pdftitle={Status of Canary Rockfish (Sebastes pinniger) along the U.S. West coast in 2023, Summary of catch values of commercial, recreational, and foreign fleets by state for state approval},
  pdflang={en},
  hidelinks,
  pdfcreator={LaTeX via pandoc}}
\urlstyle{same} % disable monospaced font for URLs
\usepackage{longtable}
% Correct order of tables after \paragraph or \subparagraph
\usepackage{etoolbox}
\makeatletter
\patchcmd\longtable{\par}{\if@noskipsec\mbox{}\fi\par}{}{}
\makeatother
% Allow footnotes in longtable head/foot
\IfFileExists{footnotehyper.sty}{\usepackage{footnotehyper}}{\usepackage{footnote}}
\makesavenoteenv{longtable}
\usepackage{graphicx}
\makeatletter
\def\maxwidth{\ifdim\Gin@nat@width>\linewidth\linewidth\else\Gin@nat@width\fi}
\def\maxheight{\ifdim\Gin@nat@height>\textheight\textheight\else\Gin@nat@height\fi}
\makeatother
% Scale images if necessary, so that they will not overflow the page
% margins by default, and it is still possible to overwrite the defaults
% using explicit options in \includegraphics[width, height, ...]{}
\setkeys{Gin}{width=\maxwidth,height=\maxheight,keepaspectratio}
% Set default figure placement to htbp
\makeatletter
\def\fps@figure{htbp}
\makeatother
\setlength{\emergencystretch}{3em} % prevent overfull lines
\providecommand{\tightlist}{%
  \setlength{\itemsep}{0pt}\setlength{\parskip}{0pt}}
\setcounter{secnumdepth}{5}
\ifxetex
  % Load polyglossia as late as possible: uses bidi with RTL langages (e.g. Hebrew, Arabic)
  \usepackage{polyglossia}
  \setmainlanguage[]{}
\else
  \usepackage[shorthands=off,main=english]{babel}
\fi

%Define cslreferences environment, required by pandoc 2.8
%https://github.com/rstudio/rmarkdown/issues/1649
\newlength{\csllabelwidth}
\setlength{\csllabelwidth}{3em}
\newlength{\cslhangindent}
\setlength{\cslhangindent}{1.5em}
% for Pandoc 2.8 to 2.10.1
\newenvironment{cslreferences}%
  {}%
  {\par}
% For Pandoc 2.11+
\newenvironment{CSLReferences}[2] % #1 hanging-ident, #2 entry spacing
 {% don't indent paragraphs
  \setlength{\parindent}{0pt}
  % turn on hanging indent if param 1 is 1
  \ifodd #1 \everypar{\setlength{\hangindent}{\cslhangindent}}\ignorespaces\fi
  % set entry spacing
  \ifnum #2 > 0
  \setlength{\parskip}{#2\baselineskip}
  \fi
 }%
 {}
\usepackage{calc}  % for \widthof, \maxof in minipage
\newcommand{\CSLBlock}[1]{#1\hfill\break}
\newcommand{\CSLLeftMargin}[1]{\parbox[t]{\csllabelwidth}{#1}}
\newcommand{\CSLRightInline}[1]{\parbox[t]{\linewidth - \csllabelwidth}{#1}\break}
\newcommand{\CSLIndent}[1]{\hspace{\cslhangindent}#1}


\providecommand{\tightlist}{%
  \setlength{\itemsep}{0pt}\setlength{\parskip}{0pt}}


\date{}
\newcommand{\trTitle}{Status of Canary Rockfish (\emph{Sebastes pinniger}) along the U.S. West coast in 2023, Summary of catch values of commercial, recreational, and foreign fleets by state for state approval}
\newcommand{\trYear}{2023}
\newcommand{\trMonth}{April}
\newcommand{\trAuthsLong}{truetrue}
\newcommand{\trAuthsBack}{Langseth, B.J., K.L. Oken}
\newcommand{\trCitation}{
\begin{hangparas}{1em}{1}
\trAuthsBack{}. \trYear{}. \trTitle{}. \glsentrylong{pfmc}, Portland, Oregon. \pageref{LastPage}{}\,p.
\end{hangparas}}

\begin{document}

%%%%% Frontmatter %%%%%

% Footnote symbols in front matter
\renewcommand*{\thefootnote}{\fnsymbol{footnote}}

\small
\thispagestyle{empty}
\pagenumbering{roman}
\noindent
\begin{center}
\title{Status of Canary Rockfish (\emph{Sebastes pinniger}) along the U.S. West coast in 2023, Summary of catch values of commercial, recreational, and foreign fleets by state for state approval}
% \textnormal{\MakeTextUppercase{\trTitle{}}}
\vspace{1.5cm}
{\Large\textbf\newline{Status of Canary Rockfish (\emph{Sebastes pinniger}) along the U.S. West coast in 2023, Summary of catch values of commercial, recreational, and foreign fleets by state for state approval}}
\vfill
by\\
Brian J. Langseth\textsuperscript{1}\\
Kiva L. Oken\textsuperscript{1}\vfill
\textsuperscript{1}Northwest Fisheries Science Center, U.S. Department of Commerce, National Oceanic and Atmospheric Administration, National Marine Fisheries Service, 2725 Montlake Boulevard East, Seattle, Washington 98112\vfill
\trMonth{} \trYear{}
\end{center}
\clearpage

% Fourth page: Colophon
\thispagestyle{empty}
\vspace*{\fill}
\begin{center}
\copyright{} \glsentrylong{pfmc}, \trYear{}\\
\end{center}
\par
\bigskip
\noindent
Correct citation for this publication:
\bigskip
\par
\trCitation{}
\clearpage

% Add TOC to pdf bookmarks (clickable pdf)
\pdfbookmark[1]{\contentsname}{toc}

% Table of contents page, lists of figures and tables
\tableofcontents\clearpage
\label{TRlastRoman}
\clearpage

% Table of contents
\newpage
\thispagestyle{empty} % to remove page number

% Settings for the main document
\pagenumbering{arabic}  % Regular page numbers
\pagestyle{plain}  % No page number on first page of main document, use 'empty'
\renewcommand*{\thefootnote}{\arabic{footnote}}  % Back to numeric footnotes
\setcounter{footnote}{0}  % And start at 1
\renewcommand{\headrulewidth}{0.5pt}
\renewcommand{\footrulewidth}{0.5pt}
%\pagestyle{fancy}\fancyhead[c]{Draft: Do not cite or circulate}

\newcommand{\lt}{\ensuremath <}
\newcommand{\gt}{\ensuremath >}

\hypertarget{total-removals}{%
\subsection{Total removals}\label{total-removals}}

Removals (1892-2022) for canary rockfish were compiled from multiple data sources. This assessment includes total removals (landings plus dead discards) by state for the following fleets: trawl, non-trawl, recreational, foreign trawl, and at-sea hake. A summary of total removals are provided in Table 5, Table 6, and Figure 2.

\hypertarget{commercial-landings-by-state}{%
\subsubsection{Commercial landings by State}\label{commercial-landings-by-state}}

\hypertarget{washington}{%
\paragraph{Washington}\label{washington}}

Commercial landings data in Washington were available 1935--2022. Historical landings (1935--1980) were retained from the last full assessment where they were assumed to be for trawl gears (see Stewart (2007) and Thorson and Wetzel (2015) for additional details). Recent landings (1981--2022) for the trawl and non-trawl (mainly hook and line) gears were obtained from \gls{pacfin}, the central repository for West coast commercial landings (extracted on 03/23/2023, Pacific States Marine Fisheries Commission, Portland, Oregon; www.psmfc.org).

\hypertarget{oregon}{%
\paragraph{Oregon}\label{oregon}}

Commercial landings data in Oregon were available 1892--2022. Historical landings from 1892 to 1986 were provided by \gls{odfw} (Karnowski, Gertseva, and Stephens 2014). Landings in 1987--1999 were compiled from a combination of \gls{pacfin}, (extracted on 03/23/2023), and a separate \gls{odfw} reconstruction that delineated canary rockfish-specific landings among unspecified species categories within \gls{pacfin} (e.g.~URCK and POP1, Oregon Department of Fish and Wildlife 2017). Canary rockfish landings from this reconstruction were substituted for the URCK and POP1 landings available from \gls{pacfin} and added to \gls{pacfin} landings from other categories for a complete time series of canary rockfish during this time period. Commercial landings in 2000--2022 were available on \gls{pacfin} (extracted on 03/23/2023).

\hypertarget{california}{%
\paragraph{California}\label{california}}

Commercial landings data in California were available 1916--2022. Historical landings from 1916 to 1980 were obtained following the process described in the southern California vermilion rockfish (\emph{Sebastes miniatus}) assessment (see E. J. Dick et al. (2021) for complete details). Briefly, landings 1916--1968 came from a state reconstruction (Ralston et al. 2010). Landings from unknown gears in known regions were allocated proportional to landings from known gears within the same region for each year. Landings from unknown locations (Region 0) and unknown gears were allocated proportional to the landings from known gears across all known locations for each year. Landings 1969--1980 came from \gls{calcom}, which for 1969--1977 incorporates fish ticket data including mixed species categories for rockfish that were assigned to individual species using the earliest species composition samples (from the late 1970s and ealy 1980s). Recent (1981--2022) landings by trawl and non-trawl gears were obtained from \gls{pacfin} (extracted on 03/23/2023).

Additional catches caught off the coast of Oregon or Washington but landed in California from 1948-1968 were included, as these were not incorporated within the (Ralston et al. 2010) reconstruction (E. Dick, personal communication, 3/16/23). These final landings are in the California landings history for consistency with treatment for other states, because historical catches landed in Oregon or Washington but caught in the other state's waters are similarly attributed to the state where they were landed.

\hypertarget{commercial-discards}{%
\subsubsection{Commercial Discards}\label{commercial-discards}}

Estimates of dead discards were combined with landed catch to provide estimates of total removals that were used as input catches within the Stock Synthesis model (Table 8). In recent years (2000--2022), estimates of dead discards were calculated from discard data, but in earlier years (before 2000), estimates of dead discards were calculated based on assumed discard ratios.

The ratio of discarded weight to landed weight (here termed the ``discard ratio'') for the domestic trawl and non-trawl fisheries was assumed at the following values: 1\% discard ratio for 1892--1980, based on the assumption that discard rates were low prior to management actions due to the high-value nature of the fishery (and consistent with the previous assessment, Thorson and Wetzel (2015)); 5\% discard ratio for 1981--1994 which is approximately the value calculated by J. Wallace, NMFS using data from (Pikitch et al., 1988); 20\% discard ratio for 1995--1999, based on the assumption that discard ratios would be higher than previous years but less than years when discard data was present due to the period of non-retention (and consistent with the previous assessment, Thorson and Wetzel (2015)).

Onboard observers were deployed on fishing vessels starting in 2002 to collect data on discarding practices. Discard amounts for the domestic trawl, non-trawl, and at-sea hake fisheries 2002--2021 were determined based on \gls{wcgop} data provided in the \gls{gemm} product (TO DO: cite GEMM). The \gls{gemm} provides estimates of dead discards coastwide. Therefore, the total coastwide estimates of dead discards in trawl and fixed gears as provided in the \gls{gemm} were allocated by state based on the total observed landings and discards for each state as observed by \gls{wcgop}. Dead discards were added to landings to obtain total removals for 2002--2021. Total removals in 2000, 2001, and 2022 where no \gls{wcgop} data were available were calculated by applying the average discard ratios from 2002--2004 (for estimates in 2000 and 2001) and 2019--2021 (for estimates in 2022) for each gear to landings for that gear.

\hypertarget{recreational-landings-and-discards-by-state}{%
\subsubsection{Recreational landings and discards by state}\label{recreational-landings-and-discards-by-state}}

\hypertarget{washington-1}{%
\paragraph{Washington}\label{washington-1}}

Washington recreational landings, in number of fish retained, were provided by \gls{wdfw} for 1967 and 1975--2022. Landings estimates 1990--2022 were reworked by the \gls{wdfw} sport sampling unit since the previous assessment. For 1968--1974, a linear ramp between the values in 1967 and 1975 was used to approximate landings. Similarly, a linear ramp between values in 1986 and 1990 were used to approximate landings 1987--1989. Recreational landings were assumed zero prior to 1967.

Discard estimates for 2002--2022, in numbers of fish released were also provided by \gls{wdfw}. Numbers of released fish by depth category were available 2005--2022 whereas total number of released fish was available 2002--2005. Released fish at unknown depths were allocated proportional to released fish at known depths in years with depth specific values. Depth-specific release-mortality rates based on agreed upon values for canary rockfish {[}Council (2014); Table 1-10{]} were used to calculate dead discards. The average total release mortality rate in 2005--2007 was used to calculate the number of dead discards for years without depth-specific releases (2002--2004).

The total discard mortality rate (release rate multiplied by release mortality rate) for 2002 was also applied to 2000--2001 landings. All three years shared a two canary daily bag limit, and were assumed to have similar discarding practices. We assumed zero recreational releases prior to 2000 when canary rockfish was included as part of a ten rockfish daily limit, and release rates were generally low (P. Anderson, personal communication, 03/20/23).

Total removals in the Washington recreational fleet were converted from numbers to weight to maintain consistency with units from other fleets. Length data for sport catches were provided by \gls{wdfw} for 1979--2022. The average length of canary rockfish by year was calculated and converted to average weight based on parameters of the length-weight relationship (\(\alpha\) = 1.040E-05 and \(\beta\) = 3.084 for length in cm and weight in kg) as used in \gls{recfin}. In years where the number of length samples was fewer than 25, lengths were borrowed from the nearest neighboring years to calculate a three year weighted-average. A value of 25 was used because it was the `small sample size' threshold used by Edward J. Dick, Edwards, and Tsou (2021) and had increased percent standard error compared to other sample sizes. Borrowing lengths from neighboring years was only applied from years with roughly similar fishing regulations. Thus, lengths were only calculated from neighboring years in the periods \textless1999 (canary rockfish part of group daily bag limits), 2000--2003 (canary rockfish with individual daily bag limits), 2004--2016 (period of no retention), and 2017--2022 (post non-retention). Any years without lengths were assigned the non-weighted-average length across years within a period. A non-weighted length was applied to give each year equal weight in calculating an overall average for each period.

\hypertarget{oregon-1}{%
\paragraph{Oregon}\label{oregon-1}}

\hypertarget{historic-ocean-boat-landings-and-discards-19792000}{%
\subparagraph{Historic Ocean Boat Landings and Discards (1979--2000)}\label{historic-ocean-boat-landings-and-discards-19792000}}

Recently, \gls{odfw} undertook an effort to comprehensively reconstruct all marine fish recreational ocean boat landings prior to 2001 (Whitman 2023, in review). Reconstructed catch estimates from the \gls{orbs} improve upon estimates from the federal \gls{mrfss}, which have known biases related to effort estimation and sampling (van Voorhees et al. 2000) that resulted in catch estimates considered implausible by \gls{odfw}. However, the \gls{orbs} sample estimates are known to lack the comprehensive spatial and temporal coverage of \gls{mrfss} Addressing this coverage issue is a major part of this reconstruction. In general, the base data and methodology for these reconstructed estimates are consistent with recent assessments for other nearshore species (Cope and Whitman 2021; Langseth et al. 2021; Taylor et al. 2021; Wetzel et al. 2021).

Prior to 2001, \gls{orbs} monitored marine species in both multi-species categories, such as rockfish, flatfish, and other miscellaneous fishes, and individual species, such as lingcod or halibut. For this comprehensive reconstruction, four species categories were selected to reconstruct, including rockfish, lingcod, flatfish and miscellaneous, which constitute the bulk of the managed marine fish species. Canary rockfish are a component of the rockfish species category.

Category-level estimates were expanded to account for gaps in sampling coverage in two separate pathways. First, estimates from five major ports were expanded to include unsampled winter months in years lacking complete coverage. Expansions were based on available year-round sampling data and excluded years where regulations may have impacted the temporal distribution of catch. Second, all other minor port estimates were expanded to include seasonal estimates in years lacking any sampling based on the amount of minor port catch as compared to all major port estimates. A subset of landings were sampled by ORBS for species compositions within these categories. Once category-level landings were comprehensive in space and time, species compositions were applied for the three multi-species categories, including rockfish, flatfish and miscellaneous fish. Borrowing rules for species compositions were specific to the category and determined based on a series of regression tree analyses that detailed the importance of each domain (year, month, port and fishing mode) to variability in compositions.

Ocean boat estimates 1979--2000 in numbers of fish of canary rockfish from the above described methods were converted to biomass using biological samples from MRFSS (A. Whitman, ODFW, personal communication, 2023). \Gls{mrfss} biological data are available 1980--1989 and 1993--2000. An annual average weight was applied to the total annual number of fish to obtain an annual biomass estimate of the landings. Several years of missing biological data (1979, 1990--1992) were filled in using neighboring years or interpolation. Landings from 1979 to 2000 averaged 29.9 mt and are of similar magnitude to the previous assessment, though landings appear to have less interannual variability than the previous assessment (Thorson and Wetzel 2015). Landings were relatively high in the first year of available data (1979) and so landings from 1972 to 1978 were initiated at zero in 1972 and linearly ramped up to the value in 1979. The provided landings do not include an estimate of discarded fish, and thus are assumed as total removals. Bag limits in the recreational fishery during this time period (prior to 2001) were generally liberal and \gls{odfw} staff recommended that no additional mortality of discarded fish be included prior to 2001.

\hypertarget{modern-ocean-boat-landings-and-discards-2001-2022}{%
\subparagraph{Modern Ocean Boat Landings and Discards (2001 -- 2022)}\label{modern-ocean-boat-landings-and-discards-2001-2022}}

Recreational landings for ocean boat modes from 2001--2022 are available from \gls{recfin}. Both retained and released estimates of mortality are included, though retained mortality contributes the vast majority to total mortality in years outside the overfished designation for canary rockfish. Release mortality is estimated from angler-reported release rates and the application of discard mortality rates from the PFMC (Council 2014). The average proportion of canary rockfish discarded averaged 4.5\% in 2001--2003 and 2017--2022. During years where canary rockfish retention was prohibited, discard rates increased to 96.1\% on average (2004--2016). From 2001 to 2022, total landings averaged 16.0 mt, ranging from 1.7 to 60.6 mt. In 2022, estimated ocean boat landings were 55.7 mt.

\hypertarget{california-1}{%
\paragraph{California}\label{california-1}}

Recreational landings in California were available 1928--2022. Historical estimates of recreational landings 1928--1979 were obtained from the previous assessment (Thorson and Wetzel 2015), which were in turn were obtained from \gls{calcom}. For years since 1979, recreational landings along with estimates of dead discards came from \gls{recfin} via \gls{mrfss} (1980--2003) and \gls{crfs} (2005--2022) sampling programs. Landings and dead discards were added together to obtain estimates of total removals. Discards prior to 1980 were assumed to be zero.

A number of years with missing or incomplete estimates for recreational removals were filled in. The removals in 1980 from \gls{mrfss} were not used due to survey quality problems related to 1980 being the first year of \gls{mrfss} (Karpov, Albin, and Van Buskirk (1995), Cope and Key (2009)). Removals for this year were averaged across removals in 1979 and 1981. No or minimal party/charter (PC) estimates were available 1993--1995, so PC mode removals for these years were calculated assuming an average ratio of private/rental (PR) to PC modes across all years with data within \gls{mrfss} and added to PR values to obtain total recreational estimates. The \gls{mrfss} sampling program did not sample 1990--1992, and so estimates during these three years were obtained as averages from neighboring years. Estimates in 1990 were the average from 1987--1989, estimates in 1991 were the average from 1987--1989 and 1993--1995, and estimates in 1992 were the average from 1993-1995. Because of the transition from \gls{mrfss} to \gls{crfs}, no estimate was available in either database in 2004 for this assessment cycle. \Gls{cdfw} provided a value of 10.59 mt for 2004 (J. Budrick, CDFW, personal communication), based on an available previous pull of \gls{mrfss} data for the gopher rockfish assessment. Estimates for California recreational removals from the previous canary rockfish assessment were not used because these appeared to only reflect estimates of landings, and not total removals.

\hypertarget{ca-recreational-data-impacted-by-covid-19-pandemic}{%
\subparagraph{CA Recreational data impacted by COVID-19 pandemic}\label{ca-recreational-data-impacted-by-covid-19-pandemic}}

The COVID-19 pandemic impacted recreational port sampling in 2020 and 2021. No sampling occurred at all in April-June, 2020. \Gls{cdfw} provided proxy values for these months (M. Parker, CDFW, personal communication). In addition, California recreational total mortality estimates in the ``rockfish genus'' were inflated due to \gls{crfs} samplers being unable to closely examine catch and identify catch to species. This was a problem for both PR and PC modes in 2020 and primarily for the PC mode in 2021. An effort was made to allocate some of the rockfish genus mortality to other rockfish species for these modes and years (J. Coates, CDFW, personal communication). An expected value of rockfish genus mortality in 2020 and 2021 was generated by mode and year according to the average proportion to the total rockfish mortality that this category represented in 2018 and 2019, when regulations were consistent with 2020 and 2021. Mortality above this expected value was attributed to the other species also based on proportions each species represented to the total from 2018 and 2019. Calculations were made by year, mode, and district. The shore-based modes were grouped in with the PR mode. Calculations were initially made in numbers of fish because rockfish genus mortality is only recorded this way. Numbers of fish by species were then converted to weight in kilograms based on average weights of fish recorded by the \gls{crfs} program by district in 2019. Total proxy values in for canary rockfish weight were summed across modes and districts and added to existing estimates for 2020 and 2021 from \gls{recfin}.

\hypertarget{foreign-catches}{%
\subsubsection{Foreign Catches}\label{foreign-catches}}

From the 1960s through the early 1970s, foreign trawling enterprises harvested considerable amounts of rockfish off Washington and Oregon, including large quantities of canary rockfish. Foreign catches of individual species were estimated by Rogers (2003) and attributed to INPFC areas 1966-1976 for canary rockfish. INPFC areas do not coincide with the spatial strata defined in this assessment. Therefore, INPFC catches were translated to strata in the following manner: catches from the US Vancouver INPFC area were designated to the 46-49°N stratum (Washington), catches in the Columbia INPFC area were allocated to the 42-46°N stratum (Oregon), and catches from the Eureka and Monterey INPFC areas were assigned to the 32-42°N stratum (California). These definitions match where the majority of commercial trawl landings occur by state for each INPFC area within \gls{pacfin}.

\clearpage

\hypertarget{references}{%
\section{References}\label{references}}

\hypertarget{refs}{}
\begin{CSLReferences}{1}{0}
\leavevmode\vadjust pre{\hypertarget{ref-cope_status_2009}{}}%
Cope, J. M., and Meisha Key. 2009. {``Status of Cabezon (\emph{{Scorpaenichthys} Marmoratus}) in {California} and {Oregon} Waters as Assessed in 2009.''} Pacific Fishery Management Council, 7700 Ambassador Place NE, Suite 200, Portland, OR 97220: Pacific Fishery Management Council.

\leavevmode\vadjust pre{\hypertarget{ref-cope2021vermillion_or}{}}%
Cope, J. M., and A. D. Whitman. 2021. {``Status of {V}ermillion Rockfish ({S}ebastes Miniatus) Along the {US} {W}est - {O}regon Coast in 2021.''} Portland, OR: Pacific Fishery Management Council.

\leavevmode\vadjust pre{\hypertarget{ref-pfmc2014status}{}}%
Council, Pacific Fishery Management. 2014. {``Status of the {P}acific Coast Groundfish Fishery: Stock Assessment Nad Fishery Evaluation.''} Portland, OR.

\leavevmode\vadjust pre{\hypertarget{ref-dick2021vermillion}{}}%
Dick, E. J., M. H. Monk, T. L. Rogers, J. C. Field, and E. M. Saas. 2021. {``The Status of Vermilion Rockfish ({S}ebastes Miniatus) and Sunset Rockfish ({S}ebastes Crocotulus) in {U}.{S}. Waters Off the Coast of {C}alifornia South of {P}oint {C}onception in 2021.''} Portland, OR: Pacific Fishery Management Council.

\leavevmode\vadjust pre{\hypertarget{ref-dick_lengthborrow_2021}{}}%
Dick, Edward J, Jason Edwards, and Tien-Shui Tsou. 2021. {``Model-Based Estimation of Average Fish Weights from Recreational Fisheries.''} \emph{Fisheries Research} 241 (106002).

\leavevmode\vadjust pre{\hypertarget{ref-karnowski_historical_2014}{}}%
Karnowski, M., V. V. Gertseva, and Andi Stephens. 2014. {``Historical {Reconstruction} of {Oregon}'s {Commercial} {Fisheries} {Landings}.''} Oregon Department of Fish; Wildlife, Salem, OR.

\leavevmode\vadjust pre{\hypertarget{ref-Karpov1995}{}}%
Karpov, K. A., D. P. Albin, and W. H. and Van Buskirk. 1995. {``The Marine Recreational Fishery in Northern {C}alifornia and Central {C}alifornia: A Historical Comparison (1958--86), Status of Stocks (1980--1986), and Effects of Changes in the {C}alifornia {C}urrent.''} \emph{California Department of Fish Game Fish Bulletin} 176.

\leavevmode\vadjust pre{\hypertarget{ref-langseth2021quillback_or}{}}%
Langseth, B. J., C. R. Wetzel, J. M. Cope, and A. D. Whitman. 2021. {``Status of Quillback Rockfish ({S}ebastes Maliger) in {U}.{S}. Waters Off the Coast of {O}regon in 2021 Using Catch and Length Data.''} Portland, OR: Pacific Fishery Management Council.

\leavevmode\vadjust pre{\hypertarget{ref-odfw2017informational}{}}%
Oregon Department of Fish and Wildlife. 2017. {``Informational Report Regarding Speciation of Unspecified Rockfish Landings in Oregon for Inclusion in Stock Assessment Time Series of Removals.''} Agenda Item {I}.2.a. March, 2017 briefing Book. Portland, OR: Pacific Fishery Management Council. \href{https://www.pcouncil.org/documents/2017/03/i2a_odfw_rpt_speciation_of_oregon_urck_and_pop1_\%0Amar2017bb.pdf/}{https://www.pcouncil.org/documents/2017/03/i2a\_odfw\_rpt\_speciation\_of\_oregon\_urck\_and\_pop1\_ mar2017bb.pdf/}.

\leavevmode\vadjust pre{\hypertarget{ref-ralston_documentation_2010}{}}%
Ralston, Stephen, Don E. Pearson, John C. Field, and Meisha Key. 2010. {``Documentation of the {California} Catch Reconstruction Project.''} US Department of Commerce, National Oceanic; Atmospheric Adminstration, National Marine.

\leavevmode\vadjust pre{\hypertarget{ref-rogers_species_2003}{}}%
Rogers, J. B. 2003. {``Species Allocation of \emph{{Sebastes}} and \emph{Sebastolobus} Species Caught by Foreign Countries Off {Washington}, {Oregon}, and {California}, {U}.{S}.{A}. In 1965-1976.''} Unpublished document.

\leavevmode\vadjust pre{\hypertarget{ref-stewart2007canary}{}}%
Stewart, Ian J. 2007. {``Status of the u.s. Canary Rockfish Resource in 2007.''} Portland, OR: Pacific Fishery Management Council.

\leavevmode\vadjust pre{\hypertarget{ref-taylor2021lingcod_n}{}}%
Taylor, I. G., K. F. Johnson, B. J. Langseth, A. Stephens, L. S. Lam, M. H. Monk, A. D. Whitman, and M. A. Haltuch. 2021. {``Status of Lingcod ({O}phiodon Elongatus) Along the Northern {U}.{S}. West Coast in 2021.''} Portland, OR: Pacific Fishery Management Council.

\leavevmode\vadjust pre{\hypertarget{ref-thorson_status_2015}{}}%
Thorson, James T., and Chantel R. Wetzel. 2015. {``The Status of Canary Rockfish (\emph{{Sebastes} Pinniger}) in the {California} Current in 2015.''} Pacific Fishery Management Council, 7700 Ambassador Place NE, Suite 200, Portland, OR 97220.

\leavevmode\vadjust pre{\hypertarget{ref-vanvoorhees2000}{}}%
van Voorhees, D., A. Hoffman, A. Lowther, W. Van Buskirk, J. Weinstein, and White J. 2000. {``An Evaluation of Alternative Estimators of Ocean Boat Fish Effort and Catch in Oregon.''} Pacific RecFIN Statistics Subcommittee.

\leavevmode\vadjust pre{\hypertarget{ref-wetzel2021copper_or}{}}%
Wetzel, C. R., B. J. Langseth, J. M. Cope, and A. D. Whitman. 2021. {``The Status of Copper Rockfish ({S}ebastes Caurinus) in {U}.{S}. Waters Off the Coast of {O}regon in 2021 Using Catch and Length Data.''} Portland, OR: Pacific Fishery Management Council.

\leavevmode\vadjust pre{\hypertarget{ref-whitman2023}{}}%
Whitman, A. D. 2023. {``Oregon Historical Marine Recreational Catch Reconstruction (1979 -- 2000).''} Salem, OR: {ODFW} Science Bulletin, Oregon Department of Fish; Wildlife.

\end{CSLReferences}
\end{document}
